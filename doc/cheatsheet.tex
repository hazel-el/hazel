\documentclass{article}

\usepackage[utf8]{inputenc}
\usepackage[T1]{fontenc}
\usepackage{booktabs}
\usepackage{amsmath}

\DeclareMathOperator{\ran}{ran}
\DeclareMathOperator{\dom}{dom}
\DeclareMathOperator{\transitive}{transitive}
\DeclareMathOperator{\reflexive}{reflexive}

\title{OWL 2 EL vs.\ standard DL Terminology Cheatsheet}
\author{Felix Distel} 
\begin{document}

\maketitle

The tables in this document aim at making it easier to adapt results from
the DL community for usage in an OWL 2 EL reasoner. The basic terminology is
detailed in Table~\ref{table:general}, the constructors used in OWL 2 EL can be
found in Table~\ref{table:constructors}, the axioms in
Table~\ref{table:axioms}.

Three points should be mentioned that should be kept it mind when translating
classical $\mathcal{EL}$ reasoning methods such as
\cite{BaaderBrandtLutz-IJCAI-05,BaaderEtAl-OWLED08DC,Sunt-09} to OWL 2 EL.
\begin{itemize}
  \item Only little literature can be found on self-restrictions in
    $\mathcal{EL}$; one example is \cite{carral2013towards}, but that is not a
    completion-based algorithm.
  \item Features for concrete domains are usually functional, whereas the
    corresponding data properties from OWL can map individuals to several data
    literals.
  \item I have not yet found DL literature about the has-key-axioms.
\end{itemize}

\begin{table}
  \caption{General Notions\label{table:general}}
  \begin{center}
    \begin{tabular}{@{}ll@{}}
      \toprule
      OWL name & DL equivalent \\
      \midrule
      class & concept \\
      class expression & concept description \\
      datatype & concrete domain, but can be treated as unary predicate \\
      data range & predicate\footnotemark[1]\\
      object property & role \\
      data property & ---\footnotemark[2]\\
      annotation property & --- \\
      individual & individual \\
      literal & concrete domain element \\
      \bottomrule
    \end{tabular}
  \end{center}
\end{table}

\begin{table}
  \caption{Constructors supported by OWL 2 EL\label{table:constructors}}
  \begin{center}
    \begin{tabular}{@{}lll@{}}
      \toprule
      OWL name & DL equivalent & DL syntax \\
      \midrule
      ObjectSomeValuesFrom & existential restriction & $\exists r. C$ \\
      DataSomeValuesFrom & ---\footnotemark[3]  & --- \\
      ObjectHasValue & --- & $\exists r. \{a\}$ \\
      DataHasValue & ---\footnotemark[3] & --- \\
      ObjectHasSelf & self restriction & $\exists r.\mathsf{Self}$ \\
      ObjectOneOf & nominals\footnotemark[4] & $\{a\}$ \\
      DataOneOf & sets of ``literal'' predicates & $(\in \{l_1, \dots, l_n\})$ \\
      ObjectIntersectionOf & conjunction & $\sqcap$ \\
      DataIntersectionOf & conjunction of predicates & $p_1 \land p_2$ \\
      \bottomrule
    \end{tabular}
  \end{center}
\end{table}

\begin{table}
  \caption{Axioms supported by OWL 2 EL\label{table:axioms}}
  \begin{center}
    \begin{tabular}{@{}lll@{}}
      \toprule
      OWL name & DL equivalent & DL syntax \\
      \midrule
      SubClassOf & concept inclusion & $C \sqsubseteq D$ \\
      EquivalentClasses & concept equivalence/definition & $C \equiv D$ \\
      DisjointClasses & concept disjointness & $C \sqcap D \sqsubseteq \bot$ \\
      \addlinespace
      SubObjectPropertyOf & role hierarchy & $r \sqsubseteq s$ \\
      SubObjectPropertyOf + ObjectPropertyChain & role inclusion &
        $r_1 \circ \cdots \circ r_n \sqsubseteq s$ \\
      EquivalentObjectProperties & role equivalence & $r \equiv s$ \\
      TransitiveObjectProperty & transitivity & $\transitive(r)$ \\
      ReflexiveObjectProperty & reflexivity & $\reflexive(r)$ \\
      ObjectPropertyDomain & domain restriction & $\dom(r) \sqsubseteq C$ \\
      ObjectPropertyRange & range restriction & $\ran(r) \sqsubseteq C$ \\
      \addlinespace
      SubDataPropertyOf & ---\footnotemark[3] & --- \\
      EquivalentDataProperties & ---\footnotemark[3] & --- \\
      DataPropertyDomain & ---\footnotemark[3] & --- \\
      DataPropertyRange & ---\footnotemark[3] & --- \\
      \addlinespace
      SameIndividual & ---\footnotemark[5] & $\{a\} \equiv \{b\}$ \\
      DifferentIndividuals & ---\footnotemark[5] &
        $\{a\} \sqcap \{b\} \sqsubseteq \bot$ \\
      ClassAssertion & concept assertion & $a \colon C$ \\
      ObjectPropertyAssertion & role assertion & $(a, b) \colon r$ \\
      DataPropertyAssertion & ---? & --- \\
      NegativeObjectPropertyAssertion, & --- & --- \\
      NegativeDataPropertyAssertion & --- & --- \\
      FunctionalDataProperty & ---\footnotemark[3] & --- \\
      HasKey & --- & --- \\
      \bottomrule
    \end{tabular}
  \end{center}
\end{table}

\footnotetext[1]{Currently, only unary predicates are supported by OWL.} 
\footnotetext[2]{This corresponds roughly to DL features, except it is not
  necessarily functional.}
\footnotetext[3]{These axioms or constructors do not have an exact equivalent
  in classical DL. The need for them arises since data properties, unlike
  features in concrete domains are not necessarily functional.}
\footnotetext[4]{OWL 2 EL only allows a single nominal, not a set of nominals.}
\footnotetext[5]{These are needed, since OWL 2 does not make the unique name
  assumption}


\bibliographystyle{plain}
\bibliography{cheatsheet}

\end{document}
